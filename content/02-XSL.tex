\chapter{XSL}

\section{Bestandteile}
Die XML Stylesheet Language besteht aus:
\begin{itemize}
\item XSLT: XSL Transformation, Sprache zur Transformation von „XML-Konstrukten“ in andere XML-Konstrukte (oder auch „Konstrukte“ in textbasierten Sprachen)
\item XPath: XML Path Language, Sprache zur Auswahl von spezifischen „XML-Konstrukten“ aus der XML-Quelldatei
\item XML-FO: XML-Formatting Objects, spezielle XML-basierte Sprache zur layoutgetreuen Ausgabe (nicht struktur- sondern designorientierte Sprache)
\item im Folgenden für uns in der Vorlesung interessant: XSLT, XPath nicht jedoch XML-FO
\end{itemize}
\vspace*{0,3cm}

Bsp. für Anwendung: XSLT zur Wandlung der abstrakten „Studis-Datei“ in eine HTML-Datei mit entsprechender Tabelle der Studis\\
\\
Wer führt die Transformation durch?
\begin{itemize}
\item \emph{standalone-Tool:} XSLprocessor (in gängigen Linux-Distributionen enthalten) Apache xalan saxon (von Michael Kay) (unterstützt auch XSLT Version 2)
\item \emph{serverseitig:} Integration der XSLT in einem WebServer, d.h. der WebServer liefert			bei Anforderung der XML-Datei bereits die mittels XSLT transformierte Datei aus!
Apache Project Cocoon
Perl-Modul AxKit
\item \emph{clientseitig:} integriert in gängige WebBrowser $\rightarrow$ Mozilla Firefox, MS Internet Explorer, Opera, Chrome, Safari können XSLT!
\end{itemize}