\chapter{XSLT}

\section{Transfomation}
Werkzeug zur Transformation von XML-basierten Daten in (meist andere) XML-basierte Daten.
\begin{lstlisting}[caption={Definition einer XML-Datei zur Transformation}, label={lst:inline-dtd}, language={XML}]
	<?xml version="1.0" ?> <!--Hinweis: Attribut encoding="UTF-8" ist bei XML default-->
	<!DOCTYPE studis SYSTEM "url/zur DTD"> //<?+<! sind Deklarationen wobei <! auf > endet.
	[KEINE DOCTYPE-Deklaration!]
	<xsl:stylesheet	
	  version="1.0" //version->Namensraumdeklaration fuer XSLT, Praefix->Postfix
	  xmlns:xsl="http://www.w3.org/1999/XSL/Transform"
	  xmlns="Namespace der Ausgabesprache, z.B. HTML" //Namensraumdeklaration fuer Ausgabesprache, Verwendung ohne Postfix und Praefix (Grund: Ersparung von Schreibarbeit)
	>
		<xsl:output method="xml" encoding ="UTF-8" //method->auch html( bitte nicht angeben!) oder text
		doctype-public"..." //Public Doctypes (Doctype definiert den HTML-Standart)
		doctype-system"url/zur/DTD" //fuer system Doctype deklaration 
		<!--Template fuer die Definition der Transformation-->
	</xsl:stylesheet>
\end{lstlisting}

\newpage
\begin{lstlisting}[caption={Transformierte XML-Datei}, label={lst:generierte-xml}, language={XML}]
  <?xml version="1.0" encoding="UTF-8" ?>
  <!DOCTYPE studis SYSTEM "url/zur/DTD">
\end{lstlisting}


Transfoframtionsvorschriften in Form von Templates (Schablonen)
\begin{itemize}
\item Templates werden nacheinander notiert, d.h. sie können nicht geschachtelt werden.
\item Templates ersetzen irgendwelche Knoten (Tags und Attribute) aus der Quelldatei.
\end{itemize}

\begin{lstlisting}[caption={Syntax einer xsl:template-Deklaration}, label={lst:xsl-template}, language={XML}]
<xsl:template match="XPath-Ausdruck">
	<!--(wohlgeformte) Ausgabe des des Templates, also Text, Tags(inklusive Attribute) und 	weitere Verarbeitungsanweisungen-->
</xsl:template>
\end{lstlisting}