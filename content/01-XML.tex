\chapter{XML}

\begin{itemize}
\item eXtensible Markup Language
\item Mittel um konkrete Auszeichnungssprachen zu definieren
\item XML selbst ist eine Metasprache, keine eigene (konkrete) Sprache
\end{itemize}
\paragraph{Auszeichnungssprache:}Sprache um reinen Text weitere Eigenschaften mitzugeben
\paragraph{Designorientiert:}Textbestandteile bekommen Aussehen (z.B. Fettdruck, rote Farbe). Beispiel: klassisches Word 1990, Druckformate (PostScript, PCL)
\paragraph{Strukturorientiert:} Struktur oder spezielle Funktion (z.B. Überschrift, Absatz, Liste, Tabelle). Beispiel: HTML, LaTeX, SGML, die meisten XML-basierten


\section{Beispiele für XML-basierte Sprachen}
\begin{itemize}
\item XHTML (auch HTML5 in XML-Variante)
\item SVG (Grafikformat)
\item XSD (Sprache zur Definition XML-basierter Sprachen)
\item Viele Konfigurationsdateien vieler Software-Pakete (z.B. Apache)
\item Dokumentformate (aktuellere) von Microsoft Office (.docx) oder Open Office
\item Austauschformate für Inhalte von relationalen Datenbanken
\end{itemize}


\section{DOCTYPE-Deklaration}
\begin{lstlisting}[caption={Syntax einer DOCTYPE-Deklaration}, label={lst:doctype-declaration}, language={XML}]
  <!DOCTYPE html(root Tag) | PUBLIC(bzw. SYSTEM, PRIVATE) "Public-Id"(bei PUBLIC) "Syst-Id"(nicht bei PRIVATE)>
\end{lstlisting}
Public-ID: ungefähr wie bei HTML “--//W3C/DTD/XHTML1.1/EN“\\
System-ID: URL, verweist auf konkrete Grammatik in Form einer DTD\\

\section{Wesentliche Eigenschaften von XML-Dateien}
Es gibt zwei wesentliche Eigenschaften, welche jedes XML-basierte Dokument erfüllen muss bzw. sollte.
\begin{itemize}
\item Wohlgeformtheit (z.B. XML-Deklaration)
  \begin{itemize}
  \item Wurzel-Tag, welcher das komplette Dokument umschließt
  \item Tags paarweise, also Start- und Endtag
  \item Korrekte Schachtelung (letzter geöffneter Tag zuerst schließen)
  \end{itemize}
\item Gültigkeit (z.B. DOCTYPE-Deklaration, insbesondere Verweis auf DTD)
  \begin{itemize}
  \item Entspricht einer konkreten Grammatik (Tagnamen, Attributnamen und Zugehörigkeit, Enthaltenseinsmodell  (Inhalt eines Tags)
  \end{itemize}
\end{itemize}


\section{Document Type Definition}
Document Type Definitionen (kurz: DTD)
\begin{itemize}
\item Definiert eine konkrete Grammatik (XML-basiert)
\item Besteht aus Text
\item Besteht nur aus Deklarationen (wegen fehlendem Wurzeltag nicht XML-basiert)
\end{itemize}

\newpage
\section{ELEMENT-Deklaration}
\begin{lstlisting}[caption={Syntax einer ELEMENT-Deklaration}, label={lst:doctype-declaration}, language={XML}]
  <!ELEMENT tagname inhaltsmodell>
\end{lstlisting}
\paragraph{tagname:}Name des Elements/Tags bestehend aus Buchstaben (Klein- und Großschreibung wird unterschieden, $<$bla$>$ ist nicht gleich $<$bLa$>$) und Ziffern, 1. Zeichen muss Buchstabe oder Unterstrich sein, theoretisch beliebig lang, praktisch kleiner 256 Zeichen, keine Umlaute verwenden.
\paragraph{inhaltsmodell:}\hspace*{0,5mm}\linebreak
\begin{tabularx}{\textwidth}{lX}
  EMPTY & (Bsp. aus XHTML: $<$!ELEMENT br EMPTY$>$ für leere Tags ohne Inhalt)\\
  (\#PCDATA) & für beliebige Zeichenfolgen (außer „\textless“, „\textgreater“, „\&“ und „"“) insbesondere keine Tags\\
   & z.B. $<$!ELEMENT title (\#PCDATA)$>$\\
  sequenz & z.B. (tagname1, tagname2) $\rightarrow$ Abfolge von tagname1 und tagname2\\
   & z.B. $<$!ELEMENT html (head, body)$>$\\
  auswahl & z.B. (tagname1 | tagname2)\\
  gemischt & ((\#PCDATA) | auswahl)*)  
\end{tabularx}
\vspace*{2mm}

Alle Inhaltsmodelle können mit nachgestellten Quantoren versehen werden:
\begin{itemize}
\item (inhaltsmodell)* beliebig oft (inkl. Keinmal)
\item (inhaltsmodell)+	beliebig oft, aber mindestens einmal
\item (inhaltsmodell)?	einmal oder keinmal
\end{itemize}
\vspace*{2mm}

Entitäten:\\
\begin{tabularx}{\textwidth}{rcX}
  \&lt & „\textless“ & Less than\\
  \&gt & „\textgreater“ & Greater than\\
  \&amp	& „\&“	& Ampersand\\
  \&quot & „"“ & Quotation mark\\
  \&auml & „ä“ & \\
  \&Auml & „Ä“ & \\	
\end{tabularx}

\newpage
\section{ATTLIST-Deklaration}
\begin{lstlisting}[caption={Syntax einer ATTLIST-Deklaration}, label={lst:attlist}, language={XML}]
  <!ATTLIST tagname attrname attrtype voreinstellung (optional) > 
\end{lstlisting}
\paragraph{attrname:}Name des Attributs, genauso aufgebaut wie Tagnamen
\paragraph{attrtype:}\hspace*{0,5mm}\\
\begin{tabularx}{\textwidth}{lX}
CDATA & beliebige Zeichenfolgen inklusive „$<$“ und „$>$“, Einschränkung bei einfachen/doppelten Anführungszeichen\\
ID & dokumentweit eindeutiger Wert, Einschränkung an Werteraum wie bei Tag- und Attributnamen! d.h. z.B. Zahlenwerte sind keine gültigen Werte vom Typ ID!	Beispiel aus HTML: $<$!ATTLIST a id ID$>$\\
IDREF & Referenz/Verweis auf einen ID-Wert, Einschränkung der Werte wie oben, aber keine 	Eindeutigkeit gefordert, da beliebig oft auf denselben ID-Wert verwiesen werden darf\\
IDREFS & beliebig viele ID-Werte, getrennt durch Leerzeichen\\
NMTOKEN(S) & „Name“, d.h. Zeichenfolge von beliebig vielen Buchstaben, Ziffern, manchen Sonderzeichen (insb. aber kein Leerzeichen), auch das erste Zeichen darf ein beliebiges Zeichen der Zahlenmenge sein bzw. mehrere Namen durch Leerzeichen getrennt\\
 & aufzaehlung: (nmtoken1|nmtoken2|nmtoken3|…) der Attributwert kann nur einer der aufgeführten „Namen“ sein\\
ENTITY & Verweis auf Entitäten $\rightarrow$ externe (auch binäre) Daten\\
ENTITIES & Verweis auf Entitäten $\rightarrow$ externe (auch binäre) Daten\\
NOTATION & Daten mit spezieller Interpretation\\
\end{tabularx}\\
\paragraph{voreinstellung:}\hspace*{0,5mm}\\
\begin{tabular}{ll}
\#REQUIRED & Pflichtattribut\\
\#IMPLIED & optionales Attribut\\
\#FIXED & wert, Attribut mit festem Wert wert\\
wert & \#IMPLIED-Attribut mit Default-Wert wert\\
$[$fehlt$]$ & \#IMPLIED-Attribut ohne Default-Wert\\
\end{tabular}

\newpage
Die „Gültigkeit“ einer XML-Datei kann anhand der DOCTYPE-Deklaration und der darin referenzierten DTD überprüft werden $\rightarrow$ mittels einem Validator z.B. für HTML-Dateien "http://validator.w3.org/".\\
Problem: Zugriff des Validators auf die DTD? Muss die DTD auf einem öffentlich zugänglichen WebSpace liegen? $\rightarrow$ NEIN, Abhilfe: Inline-DTD, siehe das Beispiel aus Listing \vref{lst:inline-dtd}.
\begin{lstlisting}[caption={Inline-DTD Beispiel}, label={lst:inline-dtd}, language={XML}]
  <?xml version="1.0" ?>
  <!DOCTYPE bla [
    <!ELEMENT ...>
      .
      .
    <!ATTLIST ...>
      .
      .
  ]>
  <bla>
      .     
      .
  </bla> 
\end{lstlisting}

\newpage
\subsection{Beispiel}
\begin{tabularx}{\textwidth}{l|l|l|l|X}
\textbf{Name}	& \textbf{Vorname}	& \textbf{Matrikelnummer}	& \textbf{Kursbezeichnung}	& \textbf{Wahlfach}\\\hline
Müller	& Max		& 012345	& TINF12B3	& WuMBasis\\
Maier	& Moritz	& 4711		& TINF12B3	& Shit\\
Schulze	& Marta		& 0815		& TINF12B5	& Gaming\\
\end{tabularx}	

\begin{lstlisting}[caption={Listeneinträge}, label={lst:inline-dtd}, language={XML}]
  <studis>
    <studi matrikelnummer="012345">
      <name nach="Mueller" vor="Max" />
      <kurs> TINF12B3 </kurs>
      <wahlfach> WuMM </wahlfach>
    </studi>
    <!-- entsprechend fuer Maier und Schulze -->
  </studis>
\end{lstlisting}

\begin{lstlisting}[caption={Baumerstellung per !ELEMENT}, label={lst:inline-dtd}, language={XML}]
  <!ELEMENT studis (studi)*>
  <!ELEMENT studi (name, kurs, wahlfach)>
  <!ELEMENT name EMPTY>
  <!ATTLIST name nach CDATA #REQUIRED attrtype evtl. NMTOKEN vor CDATA #REQUIRED attrtype evtl. NMTOKENS>
\end{lstlisting}

Anzeige abstrakter XML-Daten (nicht HTML oder SVG) im XML-fähigen WebBrowser?
\begin{itemize}
\item strukturierte Liste (mit Einschränkungen, Elemente aus- und einklappbar)
\item nicht besonders anschaulich
\item kann mit CSS deutlich „aufgehübscht“ werden
\item bessere Variante: XSLT (XML Style Sheet Language Transformation)\\
Achtung: Trotz des Namensbestandteils “Stylesheet macht eine XSLT viel mehr als nur Aussehen festzulegen!
\end{itemize}