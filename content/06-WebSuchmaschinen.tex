\chapter{Web-Suchmaschinen}


\section{Entwicklung}

\paragraph{0. Kommentierte Linklisten}\hspace*{1mm}
\begin{itemize}
\item[$\Rightarrow$] Ausnutzen des „Urprinzips“ des Web $\Rightarrow$ Hyperlink
\item hoher Pflegeaufwand
\item nur sinnvoll bei kleiner Anzahl WebSites
\item potentiell hohe Qualität der Suchergebnisse möglich
\end{itemize}

\paragraph{1. Webkataloge (Yahoo, 1994)}\hspace*{1mm}
\begin{itemize}
\item[$\Rightarrow$] Linklisten nach Kategorien sortiert
\item[$\Rightarrow$] hierarchisch geordnet
\item[-] hoher Pflegeaufwand
\item[-] benutzbar bei deutlich höherer Anzahl an WebSites, aber Grenze durch Pflegeaufwand in den am feinsten spezifischen Unterkategorien
\end{itemize}

\paragraph{2. Volltextindizierer \& WebCrawler (AltaVista, 1995)}\hspace*{1mm}
\begin{itemize}
\item[$\Rightarrow$] Volltext der WebSite wird nach allen Begriffen in DB aufgenommen und „indiziert“ (ausgenommen Trivial„begriffe“ wie z.B. „ein/eine/einer/der/die/das/...“)
\item[$\Rightarrow$] Seite wird nach links auf andere Seiten durchsucht; andere Seiten werden ebenfalls in den Index aufgenommen
\item[-] kein manueller Aufwand (außer initialer Programmierung)
\item[-] hoher maschineller Aufwand
  \begin{itemize}
  \item Rechenzeit
  \item Speicherplatz
  \item hoher Netzwerktraffic
  \end{itemize}
\item[Anmerkung:] AltaVista war ein Demonstrator der Firma DEC für die Leistungsfähigkeit seiner Rechner
\end{itemize}
Bewertung:
\begin{itemize}
\item[+] gute Aktualität
\item[+] kein manueller Aufwand
\item[+] feingranulare Ergebnisse möglich (individuelle Seiten-URL statt kompletter Websiten-URL)
\item[-] Rolle und Bedeutung der Suchbegriffe auf der Seite ist undefiniert.
\item[-] Suchmaschinen- Spamming ist möglich
\end{itemize}

\paragraph{3. Metadaten-Suchmaschine}\hspace*{1mm}
kein typischer Vertreter, sondern eingebaut in mehr oder weniger alle Web-Suchmaschinene seit Mitte der 199x-er Jahre.
\begin{itemize}
\item[$\Rightarrow$] "durchsucht" und "indiziert" werden die "Metadaten" einer Website.
\end{itemize}
\subparagraph{Einschub: Metadaten}
"Daten über Daten", d.h. Informationen und Beschreibungen über andere Daten, welche nicht selbst Bestandteil der eigentlichen Daten sind. \\
Sie sind beim Betrachten der eigentlichen Daten normalerweise unsichtbar.
\begin{itemize}
\item\underline{HTML:}<head>, und insbesondere <meta>-Tag
\item\underline{MP3:}ID3-Tags ( Komponist, Titel, Artist,...)
\item\underline{JPEG(und TIFF)} : EXIF, Aufnahmedatum, Kameramodell, Objektiv, Brennweite, mit/ohne Blitz, Ortsangaben/GPS...
\item\underline{Dateisysteme:} Name, Größe, Änderungsdatum, Besitzer, Rechte,... von Dateien
\end{itemize}
\dots praktisch (fast) alle Dateiformate \dots \\
AUSNAHMEN: Rohdaten,z.B. plain text, Folge von Audio Sampels, direkter Screen Dump,...\\
Metaddaten am Beispiel HTML
\begin{lstlisting}[caption={Metadaten am Bsp-HTML}, label={lst:metadatenHTML}, language={HTML}]
  <meta name="Rolle/Eigenschaft/proberty " content="Wert/value " />
\end{lstlisting}
$\Rightarrow$ Metadaten bestehen aus Eigenschafts-/Wert-Paaren.\\
 Zum Beispiel:\\
 \begin{lstlisting}[caption={Div. Inhalte von Name}, label={lst:beispieleNAME}, language={HTML}]
<meta name="author" content="J.W. von Goethe"/>
<!--oder-->
<meta name="auteur" content="J.W. von Goethe"/>(franzose)
<!--oder-->
<meta name="autor" content="J.W. von Goethe"/>(deutsch)
<!--oder-->
<meta name="Schriftsteller/ Verfasser" content="J.W. von Goethe"/>
\end{lstlisting}
$\Rightarrow$ Wer legt die Werte des name-Attributs (also die Rollenbezeichner) fest?\\
$\Rightarrow$ zumindest der HTML-Standart tut dies \underline{nicht!}\\
$\Rightarrow$ best-practice: z.B. author, keywords, description, titel,\dots\\
Problem: mangelnde Standardisierung und damit: Was versteht die Suchmaschine?\\
Gewünscht: Standard für Rollenbezeichner\\
Lösung: Dublin Core.\\
$\Rightarrow$ Dublin Core Element Set(DCES), u.a. standardisiert als RFC 2413 (also auch ein IETF-defacto-Standart).\\
$\Rightarrow$ breiter Einsatzbereich, aber insbesondere für Online-Zwecke.\\
$\Rightarrow$ definiert (nur) die Rollen in 3 Kategorien.\\
\newline
\begin{tabularx}{\textwidth}{|X|X|X|}
 \hline 
 Content & Interlectual Proberty & Instantion \\ 
 "Inhalt" & "Urheberschaft" & "Ausprägung" \\ 
 \hline 
 \hline
 Titel = Titel des Werkes & Creator = geistiger Urheber des Werks & Date=Datum der Erschaffung oder Veröffentlichung \\ 
 \hline 
 Subject = Thema, oft in Form von Stichwörtern & Publisher = Veröffentlicher, Herrausgeber,Verlag & Format = technisches Format, z.B. MIME-Type \\ 
 \hline 
 Description = Beschreibung, Kurzfassung & Contributer = jemand, der etwas beiträgt, aber weniger als ein Creater & Identifier = eindeutiges Identifikationsmerkmal,z.B. URL, ISBN \\ 
 \hline 
 Type = Art der Ressurce(Roman,Gedicht,...) & Rights = Urheberrechte, Hinweiß auf Kontakt o.ä. & Language = Sprache des Dokuments oder der Übersetzung \\ 
 \hline 
 Source = andere Quelle, um die im vorliegenden Wekr wesentlich geht. &  &  \\ 
 \hline 
 Relation = Rolle, welche die( in Source) andere Quelle für das vorliegende Werk spielt. &  &  \\ 
 \hline 
 Coverage = zeitliche oder räumliche Abdeckung des Werkes &  &  \\ 
 \hline 
\end{tabularx}  
DC spezifiziert \underline{nur} die Rollen (properties), nicht aber die Werte (values)! Aber für die Werte wäre auch eine Standardisierung sinnvoll.\\
Bsp.: Format einer Datumsangabe:\\
\begin{itemize}
\item 08.Dezember2014
\item Dec 08th,2014
\item 08.12.2014
\item 08.12.14
\item 12.08.14
\item 2014-12-08 (ISO)
\end{itemize}
zur Spezifikation im Meta-Tag:
\begin{lstlisting}[caption={Div. Inhalte von Name}, label={lst:beispieleNAME}, language={HTML}]
<meta name="DC.date" content="2014-12-08" scheme="YYYY-MM-DD"/>
\end{lstlisting}
Problem: Aussehen der scheme ist ebenfalls unstandardisiert (kann auch Name/Verweis auf ISO8601 sein)

\paragraph{Sprache durch Angabe im content-Attribut:} Englisch, Deutsch, Französisch, Arabisch, CHinesisch,\dots ?
\begin{lstlisting}[caption={Div. Inhalte von Name}, label={lst:beispieleNAME}, language={HTML}]
<meta name="DC.subject" lang="de-BY" content="Mia san mia."/>
\end{lstlisting}

\paragraph{anderes Bsp:}Liste von möglichen Schlüsselwörtern\\
$\Rightarrow$"Verschlagwortung" in Bibliotheken mit Hilfe de ACM Computer Classification System

\subsection{Suchmaschinen-Spamming}
Parallel zu der Entwicklungsliste\\
\underline{Ziel} : (Eigene) Website soll bei allen möglichen (häufig nachgefragten) Suchbegriffen " möglichst weit oben" in den Suchergebnissen auftauchen. 
Methoden:
\begin{enumerate}
\item Keyword Spamming: Die Begriffe, nach denen im Internet häufig gesucht wird, werden (möglichst unsichtbar oder unauffällig) auf der Website platziert.
 \paragraph{Unsichtbar?}
 \begin{itemize}
 \item kleine Schriftgröße
 \item Schriftfarbe und Hintergrundfarbe gleich.
 \item evtl. transperente Schrifft(Css3).
 \end{itemize}
$\Rightarrow$ alles \textbf{CSS-Mehtoden} !\\
Texte aus DOM(Document Objekt Model, Repräsentation von HTML/SVG-Seiten im Brwoser, welche per JavaSkript verändert werden kann.) entfernen $\Rightarrow$ \textbf{JavaSkript}
\end{enumerate}

\vspace*{1cm}
zurück zu Suchmaschinen:\\
\textbf{3.Metadaten-Suchmaschinen}\\
\textbf{Aufwand:} weniger "Volumen", also weniger Aufwand (meistens\dots) $\Rightarrow$für die Suchmaschine\\
aber: Zusatzaufwand beim Erstellen der Seite ohne unmittelbar ersichtliches "Resultat"\\
\textbf{Qualität:} klare Spezifikation der Rollen bedeutet: "Wichtigkeit" der Metadaten ist klar\\
aber: Metadaten-Spamming ist noch leichter als Keyword-Spamming\\
\textbf{PT SM-Spamming:}\\
\textbf{2.Metadaten-Spamming}\\
Angabe häufig gesuchter Texte und Begriffe in den Metadaten (welche nicht einmal versteckt werden müssen).\\
Gerichtsurteil aus Deutschland: Namen der Produkte von Mitbewerbern in den Metadaten sind unzulässig!\\
\\
\textbf{4.Meta-Suchmaschine} (MetaCrawler1995, MetaGer1996)
\begin{itemize}
\item[$\Rightarrow$]Art "Web-Oberfläche" für andere Suchmaschinen
\item[$\Rightarrow$]keine eigene Websuche, kein eigener Index,\dots
\item[$\Rightarrow$]parallele Anfrage an mehrere Suchmaschinen gleichzeitig und mit nur einer Suchanfrage-Syntax möglich
\item[$\Rightarrow$]evtl. Abgleich der verschiedenen Suchergebnisse durch die Meta-Suchmaschine möglich $\rightarrow$Mehrwert und Qualitätsgewinn möglich, sowie mögliches Ausfüllen von "Spam-Ergebnissen"
\end{itemize}
\vspace*{5mm}
\textbf{5.Suchmaschine mit Bewertungsfunktion} (Google 1998)
\begin{itemize}
\item[$\Rightarrow$]Die Seiten, welche bei einem bestimmten Suchbegriff auftauchen, bekommen eine Bewertung entsprechend \underline{Relevanz} und werden dementsprechend sortiert aufgelistet
\end{itemize}
bei Google: PageRank als Relevanz-/Güte-Kriterium\\
\(PR(i)=\frac{(1-d)}{n} + d*\sum \limits_{j}\frac{PR(j)}{c(j)}\)\\
\begin{itemize}
\item[i:] Seite, für welche der PR berechnet werden soll
\item[j:] alle Seiten, welche auf i verweisen
\item[c(j):] "count", Anzahl der Verlinkungen von j auf andere Seiten
\item[d:] "dampling", Dämpfungsfaktor \(0 \le d \le 1\)
\item[n:] Anzahl aller WebSeiten oder Websites im Netz $\Rightarrow$optional zur Normierung\\
\(\sum \limits_{k}PR(k)=1\)
\end{itemize}
$\Rightarrow$PR lässt sich gut über die Analogie zur Vererbung beschreiben\\
\textbf{Aufwand:} einiger maschineller Aufwand, da die PRs erst nach einigen Iterationen auf den jeweiligen PR konvergieren $\Rightarrow$maschinell bestimmbare "Seitengüte"\\
\textbf{Qualität:} Deutliche Verbesserung, da nach Güte sortierte Ergebnisse\\
aber: Es gibt speziell auf Google ausgerichtete Spamming-Methoden!\\
\\
PT: SM-Spam:\\
\textbf{3.Link Farming:}\\
Menge von WebSites, welche untereinander mit häufig gesuchtern Texten massiv verlinken und so gegenseitig den PR hochtreiben\\
moderne Varianten: gekaufte Backlinks, Verweise auf eigene WebSite in Kommentaren fremder Blogs oder WebForen \dots\\
$\Rightarrow$speziell auf Google zugeschnitten\\
\\
\textbf{4.Google Bombing:}\\
Stichwort "miserable failure" $\Rightarrow$ "George W. Bush"\\
Suchbegriffe werden bei Google auch mit den Linktext verknüpft, welcher auf die Seite zeigt $\Rightarrow$neben meinem PR kann auch ein Suchbegriff "vererbt" werden\\
$\Rightarrow$"Fälschen" oder Beeinflussen anderer (finder) Seiten möglich\\
\\
\textbf{5.PageCloaking:}\\
nicht speziell für Google \dots aber durch Google bekannt geworden $\Rightarrow$ BMW soll PageCloaking betrieben haben.\\
Suchmaschine bekommt andere Suchinhalte als der "normale Nutzer" ausgeliefert. Erkennung z.B. anhand des "User Agent"\\
Achtung: Übergang von "Responsive Design" zu PageCloaking ist fließend.
"Gegenbegriff" zu SM-Spamming:\\
SEO: Search Engine Optimization\\
Ziel: wie bei Suchmaschinen-Spam, aber nur bei relevanten Suchbegriffen\\
"Die guten machen SEO, die bösen machen SM-Spam!"\\
Methoden:
\begin{enumerate}
\item syntaktisch und semantisch korrektes HTML
\item gute Metadaten
\item Wichtige Begriffe (auch) am Anfang der Seite unterbringen
\item grundsätzliches Vermeiden aller Methoden, die als SM-Spamming interpretiert werden könnten
\end{enumerate}

\chapter{Klausurfragen von 2013?}
\begin{enumerate}
\item \begin{itemize}
  \item[a)]Erläutern Sie die Worte Wohlgeformt und Gültig bei einem XML-Dokument\\
  $\Rightarrow$Wohlgeformt: Nur ein Roottag, alle müssen geschlossen sein und dürfen nicht verschränkt sein (falsche Schachtelung)\\
  $\Rightarrow$Gültig: muss konkreter Grammatik entsprechen, Entsprechend zu DTD (welche Tags, Attribute, Attributwerte,\dots sind erlaubt)
  \item[b)] Beispieldatei (XML-Datenformat), welche zwei Datensätze enthält (ohne DTD)
  \item[c)] XSLT zur Ausgabe der obigen Datensätze anzeigt und sortiert\\
xsl:sort \dots mode-Attribut \dots
  \end{itemize}
\item \begin{itemize}
  \item[a)]Zwei große Gruppen von Graikformaten, Vertretern die im Web eine Rolle spielen
  \item[b)]2 Arten von Bezier Kurven in SVG. Wie werden diese beschrieben? Zusammensetzen zweier Kurven zu einer runden?
  $\Rightarrow$kubische und quadratische Bezierkurven
  \item[c)]Vollständige SVG-Datei schreiben bei der drei Bezierkurven definiert werden sollen\\
  \item[d)]copy/paste um gleichen Kurvenzug in verschiedenen Größen darzustellen $\rightarrow$ bei uns verschieben: translate=transform
\end{itemize}
\item \begin{itemize}
  \item[a)]Suchmaschinen Spamming: Begriffserklärung\\
  3Arten mit Kurzerklärung $\rightarrow$ bei uns: anders rum, 3 genannt und Erklärung gefordert!
  \item[b)]Google\\
  PR erläutern $\rightarrow$ Vererbung $\rightarrow$ bei uns: Frage nach Formel!
  \item[c)]PageRank
  \item[d)]Was bedeutet SEO und in Bezug zu a) und b) setzen
\end{itemize}
\end{enumerate}