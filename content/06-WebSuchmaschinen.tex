\chapter{Web-Suchmaschinen}


\section{Entwicklung}

\paragraph{0. Kommentierte Linklisten}\hspace*{1mm}
\begin{itemize}
\item[$\Rightarrow$] Ausnutzen des „Urprinzips“ des Web $\Rightarrow$ Hyperlink
\item hoher Pflegeaufwand
\item nur sinnvoll bei kleiner Anzahl WebSites
\item potentiell hohe Qualität der Suchergebnisse möglich
\end{itemize}

\paragraph{1. Webkataloge (Yahoo, 1994)}\hspace*{1mm}
\begin{itemize}
\item[$\Rightarrow$] Linklisten nach Kategorien sortiert
\item[$\Rightarrow$] hierarchisch geordnet
\item[-] hoher Pflegeaufwand
\item[-] benutzbar bei deutlich höherer Anzahl an WebSites, aber Grenze durch Pflegeaufwand in den am feinsten spezifischen Unterkategorien
\end{itemize}

\paragraph{2. Volltextindizierer \& WebCrawler (AltaVista, 1995)}\hspace*{1mm}
\begin{itemize}
\item[$\Rightarrow$] Volltext der WebSite wird nach allen Begriffen in DB aufgenommen und „indiziert“ (ausgenommen Trivial„begriffe“ wie z.B. „ein/eine/einer/der/die/das/...“)
\item[$\Rightarrow$] Seite wird nach links auf andere Seiten durchsucht; andere Seiten werden ebenfalls in den Index aufgenommen
\item[-] kein manueller Aufwand (außer initialer Programmierung)
\item[-] hoher maschineller Aufwand
  \begin{itemize}
  \item Rechenzeit
  \item Speicherplatz
  \item hoher Netzwerktraffic
  \end{itemize}
\item[Anmerkung:] AltaVista war ein Demonstrator der Firma DEC für die Leistungsfähigkeit seiner Rechner
\end{itemize}
Bewertung:
\begin{itemize}
\item[+] gute Aktualität
\item[+] kein manueller Aufwand
\item[+] feingranulare Ergebnisse möglich (individuelle Seiten-URL statt kompletter Websiten-URL)
\item[-] Rolle und Bedeutung der Suchbegriffe auf der Seite ist undefiniert.
\item[-] Suchmaschinen- Spamming ist möglich
\end{itemize}

\paragraph{3. Metadaten-Suchmaschine}\hspace*{1mm}
kein typischer Vertreter, sondern eingebaut in mehr oder weniger alle Web-Suchmaschinene seit Mitte der 199x-er Jahre.
\begin{itemize}
\item[$\Rightarrow$] "durchsucht" und "indiziert" werden die "Metadaten" einer Website.
\end{itemize}
\subparagraph{Einschub: Metadaten}
"Daten über Daten", d.h. Informationen und Beschreibungen über andere Daten, welche nicht selbst Bestandteil der eigentlichen Daten sind. \\
Sie sind beim Betrachten der eigentlichen Daten normalerweise unsichtbar.
\begin{itemize}
\item\underline{HTML:}<head>, und insbesondere <meta>-Tag
\item\underline{MP3:}ID3-Tags ( Komponist, Titel, Artist,...)
\item\underline{JPEG(und TIFF)} : EXIF, Aufnahmedatum, Kameramodell, Objektiv, Brennweite, mit/ohne Blitz, Ortsangaben/GPS...
\item\underline{Dateisysteme:} Name, Größe, Änderungsdatum, Besitzer, Rechte,... von Dateien
\end{itemize}
\dots praktisch (fast) alle Dateiformate \dots \\
AUSNAHMEN: Rohdaten,z.B. plain text, Folge von Audio Sampels, direkter Screen Dump,...\\
Metaddaten am Beispiel HTML
\begin{lstlisting}[caption={Metadaten am Bsp-HTML}, label={lst:metadatenHTML}, language={HTML}]
  <meta name="Rolle/Eigenschaft/proberty " content="Wert/value " />
\end{lstlisting}
$\Rightarrow$ Metadaten bestehen aus Eigenschafts-/Wert-Paaren.\\
 Zum Beispiel:\\
 \begin{lstlisting}[caption={Div. Inhalte von Name}, label={lst:beispieleNAME}, language={HTML}]
<meta name="author" content="J.W. von Goethe"/>
<!--oder-->
<meta name="auteur" content="J.W. von Goethe"/>(franzose)
<!--oder-->
<meta name="autor" content="J.W. von Goethe"/>(deutsch)
<!--oder-->
<meta name="Schriftsteller/ Verfasser" content="J.W. von Goethe"/>
\end{lstlisting}
$\Rightarrow$ Wer legt die Werte des name-Attributs (also die Rollenbezeichner) fest?\\
$\Rightarrow$ zumindest der HTML-Standart tut dies \underline{nicht!}\\
$\Rightarrow$ best-practice: z.B. author, keywords, description,, titel,\dots\\
Problem: mangelnde Standardisierung und damit: Was versteht die Suchmaschine?\\
Gewünscht: Standard für Rollenbezeichner\\
Lösung: Dublin Core.\\
$\Rightarrow$ Dublin Core Element Set(DCES), u.a. standardisiert als RFC 2413 (also auch ein IETF-defacto-Standart).\\
$\Rightarrow$ breiter Einsatzbereich, aber insbesondere für Online-Zwecke.\\
$\Rightarrow$ definiert (nur) die Rollen in 3 Kategorien.\\
\newline
\begin{tabularx}{\textwidth}{|X|X|X|}
 \hline 
 Content & Interlectual Proberty & Instantion \\ 
 "Inhalt" & "Urheberschaft" & "Ausprägung" \\ 
 \hline 
 \hline
 Titel= Titel des Werkes & Creator=geistiger Urheber des Werks & Date=Datum der Erschaffung oder Veröffentlichung \\ 
 \hline 
 Subject=Thema, oft in Form von Stichwörtern & Publisher= Veröffentlicher, Herrausgeber,Verlag & Format= technisches Format, z.B. MIME-Type \\ 
 \hline 
 Description= Beschreibung, Kurzfassung & Contributer=jemand, der etwas beiträgt, aber weniger als ein Creater & Identifier= eindeutiges Identifikationsmerkmal,z.B. URL, ISBN \\ 
 \hline 
 Type=Art der Ressurce(Roman,Gedicht,...) & Rights=Urheberrechte, Hinweiß auf Kontakt o.ä. & Language= Sprache des Dokuments oder der Übersetzung \\ 
 \hline 
 Source= andere Quelle, um die im vorliegenden Wekr wesentlich geht. &  &  \\ 
 \hline 
 Relation=Rolle, welche die( in Source) andere Quelle für das vorliegende Werk spielt. &  &  \\ 
 \hline 
 Coverage=zeitliche oder räumliche Abdeckung des Werkes &  &  \\ 
 \hline 
 \end{tabularx}  

\subsection{Suchmaschinen- Spamming}
Parallel zu der Entwicklungsliste\\
\underline{Ziel} : (Eigene) Website soll bei allen möglichen (häufig nachgefragten) Suchbegriffen " möglichst weit oben" in den Suchergebnissen auftauchen. 
Methoden:
\begin{enumerate}
\item Keyword Spamming: Die Begriffe, nach denen im Internet häufig gesucht wird, werden (möglichst unsichtbar oder unauffällig) auf der Website platziert.
\paragraph{Unsichtbar?}
\begin{itemize}
\item kleine Schriftgröße
\item Schriftfarbe und Hintergrundfarbe gleich.
\item evtl. transperente Schrifft(Css3).
\end{itemize}
$\Rightarrow$ alles \textbf{CSS-Mehtoden} !\\
Texte aus DOM(Document Objekt Model, Repräsentation von HTML/SVG-Seiten im Brwoser, welche per JavaSkript verändert werden kann.) entfernen $\Rightarrow$ \textbf{JavaSkript}

\end{enumerate}

