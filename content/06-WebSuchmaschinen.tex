\chapter{Web-Suchmaschinen}


\section{Entwicklung}

\paragraph{0. Kommentierte Linklisten}\hspace*{1mm}
\begin{itemize}
\item[$\Rightarrow$] Ausnutzen des „Urprinzips“ des Web $\Rightarrow$ Hyperlink
\item hoher Pflegeaufwand
\item nur sinnvoll bei kleiner Anzahl WebSites
\item potentiell hohe Qualität der Suchergebnisse möglich
\end{itemize}

\paragraph{1. Webkataloge (Yahoo, 1994)}\hspace*{1mm}
\begin{itemize}
\item[$\Rightarrow$] Linklisten nach Kategorien sortiert
\item[$\Rightarrow$] hierarchisch geordnet
\item[-] hoher Pflegeaufwand
\item[-] benutzbar bei deutlich höherer Anzahl an WebSites, aber Grenze durch Pflegeaufwand in den am feinsten spezifischen Unterkategorien
\end{itemize}

\paragraph{2. Volltextindizierer \& WebCrawler (AltaVista, 1995)}\hspace*{1mm}
\begin{itemize}
\item[$\Rightarrow$] Volltext der WebSite wird nach allen Begriffen in DB aufgenommen und „indiziert“ (ausgenommen Trivial„begriffe“ wie z.B. „ein/eine/einer/der/die/das/...“)
\item[$\Rightarrow$] Seite wird nach links auf andere Seiten durchsucht; andere Seiten werden ebenfalls in den Index aufgenommen
\item[-] kein manueller Aufwand (außer initialer Programmierung)
\item[-] hoher maschineller Aufwand
  \begin{itemize}
  \item Rechenzeit
  \item Speicherplatz
  \item hoher Netzwerktraffic
  \end{itemize}
\item[Anmerkung:] AltaVista war ein Demonstrator der Firma DEC für die Leistungsfähigkeit seiner Rechner
\end{itemize}