%%%%% BuildCentral
\newglossaryentry{glo:BuildCentral}{name={Buildcentral},
	description={Die \emph{BuildCentral} ist eine auf Java basierende Benutzerschnittstelle für die \emph{H/B Build Tools}, um den gesamten Bildungsprozess in einer \ac{GUI} zu steuern. Viele Teile, vor allem die Codegenerierung und Übersetzung, sind \emph{Jam} basierend. Jam ähnelt bekannten Tools wie beispielsweise make oder ant, welche benötigt werden um die Codegenerierung zu steuern.\cite{buildcentral.user.guide}}
}

%%%%% Daemon
\newglossaryentry{glo:Daemon}{name={Daemon},
	description={Als Daemon oder Dämon bezeichnet man unter Unix oder unixartigen Systemen ein Programm, das im Hintergrund abläuft und bestimmte Dienste zur Verfügung stellt. Benutzerinteraktionen finden hierbei nur auf indirektem Weg statt, zum Beispiel über Signale, Pipes und vor allem (Netzwerk-)Sockets. Der Begriff Daemon wird auch als Abkürzung von disk and execution monitor interpretiert, was jedoch ein Backronym ist.\cite{WikiDaemon}}
}


%%%%% MIDDLEWARE
\newglossaryentry{glo:Middleware}{name={Middleware},
	description={Middleware (englisch für Dienstschicht oder Zwischenanwendung) bezeichnet in der Informatik anwendungsneutrale Programme, die so zwischen Anwendungen vermitteln, dass die Komplexität dieser Applikationen und ihre Infrastruktur verborgen werden.\cite{WikiMiddleware}}
}

%%%%% MIDDLEWARE
\newglossaryentry{glo:QoS}{name={Quality of Service},
	description={Quality of Service (QoS) oder Dienstgüte beschreibt die Güte eines Kommunikationsdienstes aus der Sicht der Anwender, das heißt, wie stark die Güte des Dienstes mit deren Anforderungen übereinstimmt. Formal ist QoS eine Menge von Qualitätsanforderungen an das gemeinsame Verhalten beziehungsweise Zusammenspiel von mehreren Objekten.\cite{WikiQoS}}
}

%%%%% TCP
%\newglossaryentry{glo:TCP}{name={TCP},
%	description={Das \emph{Transmission Control Protocol} ist ein zuverlässiges und verbindungsorientiertes Transportprotokoll zur Datenübertragung in Computernetzwerken. Verbindungsorientiert bedeutet, dass vor der tatsächlichen Datenübertragung eine Verbindung zwischen Sender und Empfänger aufgebaut wird, die dann für die eigentliche Übertragung genutzt wird. Außerdem werden die empfangenen Daten automatisch auf Vollständigkeit geprüft und in die richtige Reihenfolge gebracht. Im Fehlerfall wird der Sendevorgang der fehlenden Datenpakete selbstständig wiederholt, weshalb \acs{TCP} als zuverlässig gilt.\cite{WikiTCP}}
%}

%%%%% Xtext
\newglossaryentry{glo:Xtext}{name={Xtext},
	description={Xtext ist ein Open-Source-Framework für die Entwicklung von Programmiersprachen sowie domänenspezifischen Sprachen (englisch domain specific language DSL) und ein Teil des Eclipse-Modeling-Framework-Projekts. Im Gegensatz zu normalen Parsergeneratoren wird bei Xtext nicht nur ein Parser generiert, sondern auch ein Klassenmodell für den abstrakten Syntaxbaum und ein in Eclipse integrierter Texteditor sowie die notwendige Infrastruktur für die Implementierung einer modernen Entwicklungsumgebung für die entwickelte Sprache bereitgestellt.\cite{WikiXtext}}
}
	
