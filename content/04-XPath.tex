\chapter{XPath}

\section{XPath Baumstruktur}
Sprache zur Auswahl bestimmter Knoten eines XML-Dokuments. Meist relativ zur aktuellen Position im XML-Dokument.\\
\begin{figure}[htbp]
  \includegraphics[width=\textwidth]{XPath-BaumStruktur.png}
  \caption{Baumdarstellung einer XML-Datei}
\end{figure}\\
Beschreibung der Knotensyntax:\\
„.“ $\rightarrow$ aktueller Knoten\\
„..“ $\rightarrow$ Elternknoten\\
„tagname“ $\rightarrow$ Kindelement mit "tagname"\\
„@attrname“ $\rightarrow$ Attribut mit "attrname"\\
„text()“ $\rightarrow$ Textknoten\\
„/“ $\rightarrow$ Wurzel\\
„//“ $\rightarrow$ irgendwo im Baum\\
mehrere Lokalisierungsschritte werden durch „/“ verbunden nacheinander angegeben. Bsp.: /studis/studi/name/nachname/text()\\
\\
XPath Ausdrücke liefern im allgemeinen eine Knotenmenge, d.h. mehrere Knoten (oder auch keinen)

\section{Lokalisierungsschritte}
bisher: „verkürzte Notation“\\
außerdem: ausführliche Notation\\
axis::nodetest[predicate]\\
(predicate ist optional)\\

\subsection{Achsenausdrücke (ausführliche Notation)}
\begin{tabularx}{\textwidth}{lXl}
root & Wurzelknoten & „/“\\
child & Kindknoten & „/“ {\tiny (nicht am An-}\\
 & & {\tiny fang bzw. weglassen)}\\
parent & Elternknoten & „..“\\
self & aktueller Knoten (Kontextknoten) & „.“\\
ancestor & Vorfahren, übergeordnete Knoten (Eltern, Großeltern,...)& \\
descendent & Nachkommen, untergeordnete Knoten (Kinder mit Kindeskinder) & \\
ancestor-or-self & Vorfahren inkl. Kontextknoten & \\
descendent-or-self & Nachkommen inkl. Kontextknoten & \\
following & nachfolgende Knoten (ohne Kinder und Kindeskinder des Kontextknotens) & \\
following-sibling & nachfolgende Geschwisterknoten (d.h. nachfolgende Knoten mit demselben Elternknoten wie der Kontextknoten) & \\
preceding & vorhergehende Knoten & \\
preceding-sibling & vorhergehende Geschwisterknoten (d.h. vorhergehende Knoten mit demselben Elternknoten wie der Kontextknoten) & \\
attribute & Attributknoten & „@“
\end{tabularx}

\subsection{Knotentest}
\begin{itemize}
\item Knotenname/tagname/attrname
\item „*“ als Joker für beliebige Knotennamen
\item text(), comment() für Text- bzw. Kommentarknoten 
\end{itemize}

\subsection{Prädikate}
Prädikate stehen immer in eckigen Klammern: „[Prädikatausdruck]“\\
\begin{itemize}
\item Zahl: Nummer des Knotens, Nummerierung beginnt bei 1
\item Vergleich: z.B. = [@farbe = "blau"] weitere: !=,$>$,$<$,$>=$,$<=$
\item numerische Operatoren: +,-,*,div,mod (alles Ganzzahloperatoren)
\item knotenmengen Funktionen: count (...) Anzahl der Elemente
\end{itemize}
\hspace{0,3cm}\\
zurück zu:
\begin{lstlisting}[caption={Praktisches Beispiel für xsl:template}, label={lst:template}, language={XML}]
  <xsl:template match="studis">
    <html>
      <head>
        <title>Studis an der DHBW</title>
      </head>
      <body>
        <table><xsl:applytemplates /></table>
      </body>
    </html>
  </xsl:template>
  
  <xsl:template match="studi">
    <tr>
      <td><xsl:value-of select="name/nachname/text()" /></td>
      <td><xsl:value-of select="name/vorname" /></td>
    </tr>
  </xsl:template>
\end{lstlisting}

\paragraph{xsl:value-of Syntax}\hspace{1mm}
\begin{lstlisting}[caption={xsl:value-of Syntax}, label={lst:value-of}, language={XML}]
<xsl:value-of select="XPath-Ausdruck" />
\end{lstlisting}
liefert den textuellen Wert eines Knotens bzw. einer Knotenmenge zurück
textueller Wert:
\begin{itemize}
\item eines Textknotens $\Rightarrow$ Text selbst eines Attributknotens $\Rightarrow$ Wert des anhängenden Textknotens
\item eines Elementknotens (eines „Tags“) $\Rightarrow$ Konkanetation der Werte aller Elemente und Textknoten, welche Kinder des Elementknotens sind
\end{itemize}

\paragraph{xsl:apply-templates Syntax}\hspace{1mm}
\begin{lstlisting}[caption={xsl:apply-templates Syntax}, label={lst:xsl:apply-templates}, language={XML}]
<xsl:apply-templates select="XPath-Ausdruck" />
\end{lstlisting}
\begin{itemize}
\item sucht abhängig vom Kontextknoten nach weiteren passenden Templates und führt diese aus (für Kindelemente, kann weiter eingeschränkt und auch ausgeweitet werden über optionales select-Attribut mit XPath-Ausdruck)
\end{itemize}