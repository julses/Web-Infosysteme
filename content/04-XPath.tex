\chapter{XPath}

\section{XPath Baumstruktur}
Sprache zur Auswahl bestimmter Knoten eines XML-Dokuments. Meist relativ zur aktuellen Position im XML-Dokument.\\
\begin{figure}[htbp]
  \includegraphics[width=\textwidth]{XPath-BaumStruktur.png}
  \caption{Baumdarstellung einer XML-Datei}
\end{figure}\\
Beschreibung der Knotensyntax:\\
„.“ $\rightarrow$ aktueller Knoten\\
„..“ $\rightarrow$ Elternknoten\\
„tagname“ $\rightarrow$ Kindelement mit "tagname"\\
„@attrname“ $\rightarrow$ Attribut mit "attrname"\\
„text()“ $\rightarrow$ Textknoten\\
„/“ $\rightarrow$ Wurzel\\
„//“ $\rightarrow$ irgendwo im Baum\\
mehrere Lokalisierungsschritte werden durch „/“ verbunden nacheinander angegeben. Bsp.: /studis/studi/name/nachname/text()\\
\\
XPath Ausdrücke liefern im allgemeinen eine Knotenmenge, d.h. mehrere Knoten (oder auch keinen)

\section{Lokalisierungsschritte}
bisher: „verkürzte Notation“\\
außerdem: ausführliche Notation\\
axis::nodetest[predicate]\\
(predicate ist optional)\\

\subsection{Achsenausdrücke (ausführliche Notation)}
\begin{tabularx}{\textwidth}{lXl}
root & Wurzelknoten & „/“\\
child & Kindknoten & „/“ {\tiny (nicht am An-}\\
 & & {\tiny fang bzw. weglassen)}\\
parent & Elternknoten & „..“\\
self & aktueller Knoten (Kontextknoten) & „.“\\
ancestor & Vorfahren, übergeordnete Knoten (Eltern, Großeltern,...)& \\
descendent & Nachkommen, untergeordnete Knoten (Kinder mit Kindeskinder) & \\
ancestor-or-self & Vorfahren inkl. Kontextknoten & \\
descendent-or-self & Nachkommen inkl. Kontextknoten & \\
following & nachfolgende Knoten (ohne Kinder und Kindeskinder des Kontextknotens) & \\
following-sibling & nachfolgende Geschwisterknoten (d.h. nachfolgende Knoten mit demselben Elternknoten wie der Kontextknoten) & \\
preceding & vorhergehende Knoten & \\
preceding-sibling & vorhergehende Geschwisterknoten (d.h. vorhergehende Knoten mit demselben Elternknoten wie der Kontextknoten) & \\
attribute & Attributknoten & „@“
\end{tabularx}

\subsection{Knotentest}
\begin{itemize}
\item Knotenname/tagname/attrname
\item „*“ als Joker für beliebige Knotennamen
\item text(), comment() für Text- bzw. Kommentarknoten 
\end{itemize}

\subsection{Prädikate}
Prädikate stehen immer in eckigen Klammern: „[Prädikatausdruck]“\\
\begin{itemize}
\item Zahl: Nummer des Knotens, Nummerierung beginnt bei 1
\item Vergleich: z.B. = [@farbe = "blau"] weitere: !=,$>$,$<$,$>=$,$<=$
\item numerische Operatoren: +,-,*,div,mod (alles Ganzzahloperatoren)
\item knotenmengen Funktionen: count (...) Anzahl der Elemente
\end{itemize}
\hspace{0,3cm}\\
zurück zu:
\begin{lstlisting}[caption={Praktisches Beispiel für xsl:template}, label={lst:template}, language={XML}]
  <xsl:template match="studis">
    <html>
      <head>
        <title>Studis an der DHBW</title>
      </head>
      <body>
        <table><xsl:applytemplates /></table>
      </body>
    </html>
  </xsl:template>
  
  <xsl:template match="studi">
    <tr>
      <td><xsl:value-of select="name/nachname/text()" /></td>
      <td><xsl:value-of select="name/vorname" /></td>
    </tr>
  </xsl:template>
\end{lstlisting}

\paragraph{xsl:value-of Syntax}\hspace{1mm}
\begin{lstlisting}[caption={xsl:value-of Syntax}, label={lst:value-of}, language={XML}]
<xsl:value-of select="XPath-Ausdruck" />
\end{lstlisting}
liefert den textuellen Wert eines Knotens bzw. einer Knotenmenge zurück
textueller Wert:
\begin{itemize}
\item eines Textknotens $\Rightarrow$ Text selbst eines Attributknotens $\Rightarrow$ Wert des anhängenden Textknotens
\item eines Elementknotens (eines „Tags“) $\Rightarrow$ Konkanetation der Werte aller Elemente und Textknoten, welche Kinder des Elementknotens sind
\end{itemize}

\paragraph{xsl:apply-templates Syntax}\hspace{1mm}
\begin{lstlisting}[caption={xsl:apply-templates Syntax}, label={lst:xsl:apply-templates}, language={XML}]
  <xsl:apply-templates select="XPath-Ausdruck" />
\end{lstlisting}
\begin{itemize}
\item sucht abhängig vom Kontextknoten nach weiteren passenden Templates und führt diese aus (für Kindelemente, kann weiter eingeschränkt und auch ausgeweitet werden über optionales select-Attribut mit XPath-Ausdruck)
\item rekursiver Aufruf
\end{itemize}

\paragraph{xsl:for-each Syntax}\hspace{1mm}
\begin{lstlisting}[caption={xsl:for-each Syntax}, label={lst:xsl:for-each}, language={XML}]
  <xsl:for-each select="XPath-Ausdruck">
    <!-- Block, welcher fuer jeden Knoten in der durch den XPath-Ausdruck bestimmten Knotenmenge ausgegeben wird - relative Position im Dokumentbaum entspricht diesem Knoten -->
  </xsl:for-each>
\end{lstlisting}
\begin{itemize}
\item Schleife $\rightarrow$ iterativer Aufruf
\end{itemize}

\paragraph{xsl:if Syntax}\hspace{1mm}
\begin{lstlisting}[caption={xsl:if Syntax}, label={lst:xsl:if}, language={XML}]
  <xsl:if select="XPath-Ausdruck">
    <!-- Block, welcher ausgegeben wird, falls Bedingung wahr ist -->
    <!-- Achtung: Es gibt kein else! -->
  </xsl:if>
\end{lstlisting}
Bedingung: Wie bei Prädikaten\\
z.B. einfacher XPath-Ausdruck $\Rightarrow$ wahr, falls Knotenmenge nicht leer.\\
„echte“ Bedingungen, z.B. xpath1 \&lt xpath2 (oder \&gt oder $=$ oder $!=$ oder \&lt= oder \&gt=).

\paragraph{xsl:choose Syntax}
$<$xsl:choose$>$ ist ähnlich dem switch-case normaler Programmiersprachen.
\begin{lstlisting}[caption={xsl:choose Syntax}, label={lst:xsl:choose}, language={XML}]
  <xsl:choose>
    <xsl:when test="bedingung1">
      <!-- Ausgabe falls Bedingung1 wahr ergibt -->
    </xsl:when>
    <xsl:when test="bedingung2"> ..... </xsl:when>
    <!-- beliebig viele xsl:when moeglich -->
    
    <xsl:otherwise>
      <!-- Ausgabe, falls keine der vorigen Bedingungen wahr ergibt.
      Otherwise ist Optional -->
    </xsl:otherwise>    
  </xsl:choose>
\end{lstlisting}


\paragraph{xsl:sort Syntax}
Innerhalb von xsl:apply-templates> und <xsl:for-each>
\begin{lstlisting}[caption={xsl:sort Syntax}, label={lst:xsl:sort}, language={XML}]
  <xsl:sort select="XPath" (oder="descending", default=ascending) [data-type="number" (default=text)] />
\end{lstlisting}
$<$xsl:sort$>$ ist das erste Kind von $<$xsl:for-each$>$ (bzw. einziges Kind von $<$xsl:apply-templates$>$)\\
auch mehrere $<$xsl:sort$>$ unmittelbar hinter einander sind möglich für Mehrfachsortierung (bei Gleichheit des vorigen Sortierkriteriums relevant)

\paragraph{Benamte xsl:template Syntax}
Innerhalb von $<$xsl:apply-templates$>$ und $<$xsl:for-each$>$
\begin{lstlisting}[caption={xsl:template mit name Syntax}, label={lst:xsl:template_with_name}, language={XML}]
  <xsl:template name="bla">
    <!-- irgendeine Ausgabe -->
  </xsl:template>
\end{lstlisting}
Diese benamten Templates können überall (in jedem Template) aufgerufen werden. Die Syntax dazu ist in Listing \vref{lst:xsl:call-template} aufgeführt.\\
\begin{lstlisting}[caption={xsl:call-template Syntax}, label={lst:xsl:call-template}, language={XML}]
  <xsl:call-template name="bla" />
\end{lstlisting}

\paragraph{normale Templates}\hspace{1mm}
\begin{lstlisting}[caption={Normale xsl:template Syntax}, label={lst:xsl:template_normal}, language={XML}]
  <xsl:template match="XPath" [mode="fasel"]>
  
  </xsl:template>
\end{lstlisting}
\begin{itemize}
\item werden per $<$xsl:apply-templates$>$ nur dann aufgerufen, wenn dieses ebenfalls ein mode-Attribut mit demselben Wert hat!
\item damit sind mehrfache Templates für denselben Knoten und damit auch mehrfache Durchläufe durch den Knotenbaum mit verschiedenen Ausgaben möglich
\end{itemize}

\paragraph{Variable?}\hspace{1mm}
\begin{lstlisting}[caption={xsl:variable Syntax}, label={lst:xsl:variable}, language={XML}]
  <xsl:variable name="meine_var" select="wert" />
\end{lstlisting}
\begin{itemize}
\item Der Variablenwert kann nur einmal gesetzt und nicht mehr verändert werden
\item Wert der Variablen kann in jedem XPath-Ausdruck mit \$meine\_var verwendet werden
\item Verwendung z.B. zum „Zwischenspeichern“ eines Wertes abhängig von der Knotenposition und Wiederverwenden auch dann, wenn die Knotenposition verändert wurde
\end{itemize}
\begin{lstlisting}[caption={Beispiel für Variablendeklaration}, label={lst:xsl:variable}, language={XML}]
  <bla wert="fasel">
    <blubb wert="fas" />
    <blubb werd="el" />
    <blubb wert="fasel" />
  </bla>
\end{lstlisting}
Bearbeite nur die Knoten blubb, deren Attribut „wert“ denselben Wert hat wie das Attribut „wert“ des Elternknotens bla!\\
\begin{lstlisting}[caption={Beispiel für Variable in Template}, label={lst:xsl:variable_in_template}, language={XML}]
  <xsl:template match="bla">
    <xsl:variable name="blawert" select="@wert" />
    <xsl:for-each select="blubb[@wert=$blawert]">
      <!-- Wir haben einen blubb-Knoten mit identischem Wert wie der bla-Knoten! -->
    </xsl:for-each>
  </xsl:template>
\end{lstlisting}

Ausgabe ins Zieldokument:
\begin{itemize}
\item direkte Übernahme von Tags, Texten, Werten von Attributen ins Zieldokument, wie im Template notiert\\
aber: Whitespace wird auf ein Trennzeichen reduziert!
\item Text innerhalb von $<$xsl:text$>$ $\Rightarrow$ Whitespace bleibt erhalten
\end{itemize}

\begin{lstlisting}[caption={Beispiel für Variable in Template}, label={lst:xsl:variable_in_template}, language={XML}]
  <xsl:template match="bla">
    <fuba>
      <xsl:attribute name="automobil">
        <xsl:value-of select="@wert" />
      </xsl:attribute>
    </fuba>
  </xsl:template>
\end{lstlisting}

\paragraph{xsl:attribute Syntax}\hspace{1mm}
\begin{lstlisting}[caption={xsl:sort Syntax}, label={lst:xsl:sort}, language={XML}]
  <xsl:attribute name="aname">
    <!--Wert des Attributs -->
  </xsl:attribute>
\end{lstlisting}
\begin{itemize}
\item Gibt dem unmittelbar vorher erzeugten Elemntknoten (und noch offenem Knoten) einen Attributknoten „@aname“ mit Wert „Wert des Attributs“
\end{itemize}

\section{Übung}
http://dh.jroethig.de $\rightarrow$ WuMMbasIS\\
$\Rightarrow$XML und XSLT