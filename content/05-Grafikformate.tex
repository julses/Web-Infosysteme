\chapter{Grafikformate im Web}
\begin{table}[ht]
\begin{tabularx}{\linewidth}{>{\parskip1ex}X@{\kern4\tabcolsep}>{\parskip1ex}X}

\hfil\bfseries Rastergrafiken
&
\hfil\bfseries Vektorgrafiken
\\\cmidrule(r{3\tabcolsep}){1-1}\cmidrule(l{-\tabcolsep}){2-2}

%% PROS, seperated by empty line or \par
Aufteilung der Darstellungsfläche in meist gleichgroße und quadratische Teilflächen.\par
Zuweisung (Attributierung) von Farbwerten, Helligkeitswerten, Transparenzwerten zu den einzelnen Teilflächen.\par
\textcolor{green}{\large \textbf{$+$}} einfache Darstellung auf herkömmlichen rasterbasierten Ausgabegeräten (LCD, Laser, Tintenstrahldrucker)\par
\textcolor{green}{\large \textbf{$+$}} einfache Übernahme von natürlichen Darstellungen möglich\par
\textcolor{red}{\large{\textbf{$-$}}} Skalierbarkeit schlecht: Vergrößerung des Rasters

&

%% CONS, seperated by empty line or \par
Etablieren eines Koordinatensystems auf der Darstellungsfläche\par
Darstellung von geometrischen Grundformen.\par
Attributierung der Grundformen mit Position, Größe, Farbe, Helligkeit, Transparenz, Muster, Strichdaten, ... (letztere optional)\par
\textcolor{green}{\large \textbf{$+$}} nahezu unbegrenzte Skalierbarkeit\par
\textcolor{green}{\large{\textbf{$+$}}} auch vektorbasierte Ausgabegeräte verfügbar: Plotter, vektorbasierte CRT-Bildschirme\par
\textcolor{red}{\large{\textbf{$-$}}} Rendering ist recht aufwändig

\end{tabularx}
\end{table}

\section{Rasterbasierte Grafikformate}
Zuweisung von Farbe $\rightarrow$ Farbmodell: RGB (Rot+Grün+Blau)\\
Helligkeit ergibt sich durch die Farbwerte: größere Werte $\rightarrow$ mehr Helligkeit.\\
Üblicherweise wird jedem Farbanteil ein 8bit-Wert zugewiesen $\Rightarrow$ ein Bildpunkt benötigt $3*8bit=3Byte$\\
Bsp: Digitalkamera mit 25Megapixel $\Rightarrow$ 75MByte Speichervolumen pro Bild, 16GByte Speicherkarte $\Rightarrow$ „nur“ 200 Bilder $\Rightarrow$ Kompression der Speicherdaten!!!\\

\subsection{GIF}
Graphics Interchange Format wurde in den 1980er Jahren erfunden und verwendet das RGB-Farbmodell. Das Format ist für computergenerierte schematische Darstellung mit großen einfarbigen Flächen gut geeignet.
\paragraph{Manko}zu Beginn wird eine Farbtabelle mit maximal 256 Farben definiert und fortan jede Teilfläche nur noch mit einem 8bit Wert als Index auf diese Tabelle kodiert
\begin{itemize}
\item[$\Rightarrow$]Beschreibung der gleichzeitig verwendeten Farbenzahl auf 256 Farben (aus insgesamt $\approx$ 16Mio möglichen Farben)
\item[$\Rightarrow$]Reduktion der Datenmenge durch Verlust an Information (welcher klar sichtbar ist). Anschließende Kompression der Folge von Indexwerten: modifizierte Lauflängenkodierung (Wert und Anzahl)
\item[$\Rightarrow$]funktioniert bei häufig aufeinanderfolgenden Werten
\item[$\Rightarrow$]verlust\underline{freies} Kompressionsverfahren bei GIF!!!
\item[$\Rightarrow$]aber Voraussetzung für das Funktionieren des Kompressionsverfahrens ist die vorherige Reduktion auf 256 Farbwerte
\item[$\Rightarrow$]Wahrscheinlichkeit gleicher aufeinanderfolgender Werte damit recht hoch!
\end{itemize}
\paragraph{Transparenz} eine Farbe kann als transparent markiert werden, jedoch ist keine Teiltransparenz möglich.
\paragraph{Animated GIF}erlaubt kurze Sequenzen von Bildern, welche unmittelbar nacheinander dargestellt werden $\Rightarrow$ „Filmeffekt“

\subsection{JPEG}
Joint Photographic Expert Group verwendet das RGB-Farbmodell oder (alternativ) CMYK (für Druck)
\begin{itemize}
\item[$\Rightarrow$]keine Transparenz möglich
\item[$\Rightarrow$]keine Reduktion der Farbanzahl, also 16Mio. (bei RGB) bzw. 4Mrd. (bei CMYK) Farben gleichzeitig (theoretisch) möglich
\end{itemize}
\newpage
\subsubsection{DCT: Discrete Cosine Transform}
\begin{itemize}
\item[a)] Mittelwert über gesamtes (zunächst 8x8 Pixel) Areal
\item[b)] Halbierung des (8x8 Pixel-) Blockes in x- und y-Richtung auf vier (4x4 Pixel-) Blöcke
\item[c)] Berechnung der Mittelwerte dieser kleineren Blöcke
\item[d)] Abspeichern als Differenzen der kleinen Blöcke zum großen Block
\item[e)] weiter bis 1x1 Pixel-Block erreicht
\end{itemize}
DCT sorgt für denselben Informationsgehalt bei gleicher Datenmenge (keine Kompression!)\\
"Qualitätseinstellung" durch Zusammenfassen von Koeffizienten mit ähnlichen Werten auf den gleichen Wert $\Rightarrow$ Informationsverlust, hilft der Effizienz des nachfolgenden Huffman

\paragraph{Huffman-Codierung: verlustfrei}abhängig von der Häufigkeit ihres Vorkommens werden die zu kodierenden Werte mit kürzeren oder längeren Codewerten dargestellt $\Rightarrow$ variabel lange Codeworte für die Werte!\\
Prinzip: Binärbaum\\
Beispiel:
\begin{table}[h]
\begin{tabular}{rcl}
0 & $\Rightarrow$ & fertig \\
1  & $\Rightarrow$ & kommt noch was\\
"0"  & $\Rightarrow$ & für häufigsten Wert\\
"10"  & $\Rightarrow$ & für zweithäufogsten Wert\\
"110" & $\Rightarrow$ & für dritthäufigsten Wert\\
bis zu "11... 10" (255 Einsen) & $\Rightarrow$ & für seltensten Werte
\end{tabular}
\end{table}\\
$\Rightarrow$funktioniert ganz gut bei DCT-Koeffizienten ("Abweichungswerten"), da die Abweichungen häufig kleine Werte darstellen $\Rightarrow$ kurze Codelänge für diese kleinen Werte\\
$\Rightarrow$verlustfreie Komprimierung!

\subsubsection{Nachfolgeformat: JPEG2000}
\begin{itemize}
\item[$\Rightarrow$] unterstützt RGBA (zusätzlicher Alpha-Kanal/Transparenz mit 8bit)
\item[$\Rightarrow$] nur mit Lizenzgebühren zu verwenden
\item[$\Rightarrow$] keine weite Verbreitung bislang
\end{itemize}

\newpage

\subsection{PNG}
Portable Network Graphics wurde in den 1990er Jahren als Ersatz für GIF entwickelt (wg. Patentproblematik)
\paragraph{2 Modi:} \begin{itemize} \item indizierte Farben wie bei GIF
\item unreduzierte RGBA-Farben\end{itemize}
$\Rightarrow$beliebige (auch Teil-)Transparenz möglich. In jedem Fall verlustfreie Kompression nachgeschaltet.\\
$\Rightarrow$für photorealistische Darstellungen of deutlich größeres Dateivolumen als bei JPEG! Dafür evtl. JNG $\rightarrow$ keine große Verbreitung.